\documentclass{scrartcl}
\usepackage[utf8]{inputenc}
\usepackage{indentfirst}
\setcounter{secnumdepth}{0}

\title{Poupon: Technical Report, Phase 1}
\subtitle{CS 373 Software Engineering, Fall 2017}
\author{Sarah Wang, William Xu, Logan Zartman, Christian Carroll, Zach Burky}
\date{September 28, 2017}

\begin{document}

\maketitle

\tableofcontents

\newpage
\section{Motivation}
\indent
Music has always been a dominant force in the past centuries. Encompassing many different genres and cultures, music has provided another means in which others can learn about themselves or others. While the music industry has experienced a 40\% decline globally in the last two decades, with the release of portable audio players and streaming services such as Pandora, Spotify, and Apple Music, the music industry has seen amazing growth. With a nearly 15 billion and rising global revenue this past year, the music industry’s revival seems all but impossible. Additionally, as the Information Age and Internet of Things (IoT) phenomenon comes into full swing, we’ve seen the music industry blossom as never been seen before.

With streaming services in full bloom, it’s rare to meet a classmate, friend, or family member who hasn’t heard of Spotify, Apple Music, Pandora, or any of the major music streaming companies. In fact, streaming services are so intertwined in this age with the modern music industry that many up and coming artists have had their first albums, mixtapes, and songs hosted on one or more of these platforms (such as BandCamp or SoundCloud). Gone are the original days of getting the attention of producers and record labels through forced encounters or even snail mail, in many ways, like many other IoT and Data companies, the music industry has evolved from a previously offline process to a very entrenched, online process. Viral videos and views are the way that much of today’s stars have had their humble beginnings. Additionally, shows like American Idol and The Voice have used power of mobile phones to full use, combining the IoT phenomenon that we live in to discovering new stars that we follow ferverously.

Given the vast number of genres available at our disposal through streaming sites, it’s difficult to understand why we would decide to home in or focus on only hip-hop/R\&B. However, for many of us, these two genres encompassed much of our childhood and was a solid foundation for many of the youth in our generation. Songs from Usher’s earlier albums like Confessions to even Kendrick Lamar’s newest album DAMN., have played anywhere from the crowded cities of New York to the quiet suburbs of cities like Plano. Additionally, hip-hop/R\&B’s rich intertwined history in our country’s own history naturally makes it a great genre to focus on and potentially present to others. All in all, hip-hop/R\&B’s ability to connect others despite background, social standing, and culture makes it an excellent medium to focus on for our app. It’s ability to transcend boundaries and relatability make for a great conversation starter, providing a great avenue for others to meet each other.

On an related note, the name of our domain, poupon.me, comes from the dijon mustard, Gray Poupon, which has been mentioned in numerous popular songs in the past decades by artists like Kanye West and Jay-Z. The brand’s status of luxury, style, and class as well as its easiness to rhyme make it a natural inclusion in most songs. For us, it was no question to incorporate it into our site, built to spread, educate, and introduce Hip-Hop and R\&B to others.

\section{Use Cases}
\indent As mentioned earlier, poupon.me’s primary use case is to introduce Hip-Hop and R\&B to all listeners. Whether they’re new listeners or “experienced” listeners, our website is aimed at helping listeners experience Hip-Hop in a different way than the “shuffle” button on streaming services such as Spotify. Our site is dedicated to seeing Hip-Hop and R\&B in different ways; By classifying artists by their most successful albums, city, or even their top acclaimed songs, we can introduce a more introspective way of listening. For example, by viewing Artists and even Albums by Cities, we can help listeners find related Artists or even styles, dividing cities into different sub-genres much like the “East Coast vs West Coast” rivalry of the mid to late 1990s. The news tab also can help introduce the newest, trending songs to listeners, without the need of going through other websites or being bombarded by their friends. In short, our website provides an additional view of listening to hip-hop and R\&B on top of the original methods of scrolling through songs and albums. 

While our original attention is for poupon.me to contain information on only Hip-Hop and R\&B, the structure of the API allows for other genres and even single albums, artists, cities, and articles to be added without any worries.

\section{RESTful API}
The first API that we used was the Spotify API: \begin{verbatim}https://developer.spotify.com/web-api/endpoint-reference/\end{verbatim}

While intimidating at first, making use of the Spotify API became much easier as we familiarized ourselves with Spotify’s Web API over the course of the week. Spotify provides excellent documentation for its own API and has a wide variety of different calls that we can call to receive the information we need. In fact, there were some calls like “GET Audio Analysis for Track” that provided some ideas for future iterations on top of our site. All in all, Spotify’s API was an excellent resource for pulling Album, Artist, and Song information and gave us inspiration for potential additions onto our site.

Although the API was well documented and simple, our first major problem with working with the Spotify API was acquiring an authorization code/access token for using the API. Thankfully, Spotify had an authorization guide that we followed in order to resolve this. For generating our API requests, We debated over using Flask, cURL, or a third party GUI like Postman or the Insomnia REST Client. For now, the team decided to use a third party GUI when creating the persistence layer, but we did not decide on which specific app to use yet. Additionally, the team wanted to use Flask to serve additional requests to potentially pull new data points that were not in our persistence layer. Since the persistence layer is not necessary for the first sprint, we decided to use a combination of cURL and the provided web tool in Spotify’s developer portal and finalize the structure of the persistence layer at a later iteration of our project. 

Our second problem with using the API, was the fact that Spotify stored much of their information through unique IDs. Rather than being able use an API call with an artist’s name, we had to first use the search-item endpoint to determine the unique ID, and then use another API call to get the information that we needed.

As of now, the API Calls specifically used were the following:

\begin{itemize}
    \item This is the API Call that we use to search for the unique IDs of each of the following fields:
    \begin{verbatim}GET https://api.spotify.com/v1/search\end{verbatim}

    \item This is the API Call to retrieve album information from the API:
    \begin{verbatim}GET https://api.spotify.com/v1/albums/{id}\end{verbatim}

    \item This is the API Call to retrieve artist information from the API:
    \begin{verbatim}GET https://api.spotify.com/v1/artists/{id}\end{verbatim}

    \item This is the API Call to retrieve related artist information:
    \begin{verbatim}GET https://api.spotify.com/v1/artists/{id}/related-artists\end{verbatim}
\end{itemize}

\noindent
The second API that we used was the PRAW Reddit API:
\begin{verbatim}https://github.com/reddit/reddit/wiki/API\end{verbatim}

\noindent
We used the PRAW (Python Reddit API Wrapper) in order to easily scrape data about articles:
\begin{verbatim}http://praw.readthedocs.io/en/v3.6.1/pages/getting\_started.html\end{verbatim}

So far, we used the wrapper to write a script to pull a few Reddit posts to serve as examples. We wrote a script that went to the reddit.com/r/hiphopheads top posts of all time. The script then scraped articles that related to the artists we have on our website at this point.

In the future we will explore whether we want to use the Reddit API directly or continue using PRAW. We will need to pull many more articles and we will have many more artists in our database, which may make it easier to just use the Reddit API.\\

\noindent
The third (and final so far) API that we will use is the US Census Bureau API:
\begin{verbatim}https://api.census.gov/data.html\end{verbatim}

This implementation is more vague, as the sheer amount of data sets that the Census contains is abnormally large, and the amount that we want to take from it (accurate population values, as of right now) is very limited, so this is a TODO for now.

\section{Models}
We have chosen four models to use in our site: artists, albums, articles, and cities.

\subsection{Artist}
An artist is a person or group that has released music. This model contains personal information about an artist as well as information about the music they have released. This information is scraped using the Spotify API.\\

Attributes:
\begin{itemize}
    \item Name
    \item Genres
    \item Albums
    \item Image
    \item Related Artists
\end{itemize}

If we desired, there’s obviously any number of attributes that could be added to an artist - cities, trending news, age, photographs, embedded media, links to songs/albums, etc. However, in the earliest iteration of the site we are keeping it so that the amount of attributes connected to any given node is minimal, so as to avoid confusion when learning how to integrate all the different pieces that combine to make a functional stack. Once we have a more functional working product (and practical understanding of web dev), it will be a lot more feasible to consider expanding the models’ attributes. 

\subsection{Album}
An album is a collection of songs released as a group. This model contains a list of songs as well as metadata about the album. This information is scraped using the Spotify API.\\

Attributes:
\begin{itemize}
    \item Name
    \item Artists
    \item Release Date
    \item Label
    \item Cover Art
    \item Tracklist
\end{itemize}

\subsection{Article}
An article contains information about a hip hop related news article and the artists mentioned in the article. We pull our articles from reddit.com/r/hiphopheads using the Reddit API.\\

Attributes:
\begin{itemize}
    \item Title
    \item Score (Reddit upvotes)
    \item Number of comments
    \item URL
    \item Artists
\end{itemize}

\subsection{City}
The city model contains general information about a city in addition to hip hop specific information. This information is pulled from the census data.\\

Attributes:
\begin{itemize}
    \item Name
    \item State
    \item Population
    \item Artists
    \item Coordinates
\end{itemize}

\section{Hosting}
Our web app is hosted on Google Cloud Platform. We use Google App Engine to simplify the process of running our Flask backend and serving the static files for our React web app. We use the GCP web console to manage our app and the GCP command line tools to deploy it from our local machines. We create a small app.yaml configuration file to specify what runtime we need (Python) and how to route requests to our website. Requests are routed either to the backend if they begin with /api/ or otherwise to our static file directory (which contains the React web app).

We acquired our domain via the educational license on Namecheap. In order to connect it with our GCP App Engine, we first had to get GCP to assign it a configuration. From there, we were able to edit the records inside the poupon.me domain, including adding four A level Records, 4 AAAA level Records, a TXT Record, and a CNAME Record. From there, GCP was able to integrate itself with our domain.

\section{Tools}

\subsection{make}
GNU Make enables us to create recipes for simple automated builds of source and non-source files. Make recipes are defined in a Makefile in the top level directory of our repository. We handle packaging our web app with the Webpack setup provided by create-react-app. We do, however, use make to allow us to easily deploy to Google Cloud Platform, test our app locally, and even install necessary build tools.

\subsection{git}
We use a git repository hosted on GitHub to provide version control. Using a centralized version control system enables us to work in parallel on the same files and merge our edits—often automatically. Hosting our project in a git repository also allows us to revert changes that cause new bugs, or even inspect older versions of the code on our machines. Git ensures that we never lose work, and never make destructive edits.

\subsection{Bootstrap}
Bootstrap is a CSS framework that provides a “responsive” layout system—that is, the layout scales and rearranges automatically to fit various screen sizes and device types. Bootstrap also helps to normalize page styles across various browsers, and provides a selection of components that can be used in our web app. Components include things such as buttons, dialogs, navigation bars, and cards for grouping content. Bootstrap allows us to design a responsive layout by setting up a 12-column grid, and specifying how many columns should be occupied by a given element on a device with a given screen size. For example, a card might have a width of 4 columns on large displays but the full 12 columns on small displays. A full-width banner, on the other hand, would have a size of 12 columns on all displays.

\subsection{React}
React is a component-based, declarative JavaScript framework for building UIs. React components are essentially templates that allow the programmer to render data into HTML to present to the user. This data can be retrieved dynamically and be used to update components entirely on the client side—in our setup, no rendering occurs on the server. However, React is distinct from some template frameworks in that React components are written in a superset of JavaScript called JSX. JSX enables HTML templates and JavaScript logic to be written in the same file. This means that there is no unnecessary shoehorning of data and logic into HTML-based templates.

\subsection{create-react-app}
create-react-app automates the process of creating the skeleton for a React web app. Additionally, it automatically configures Webpack to build an optimized production version of our React website. create-react-app enables us to use JSX and ES6 syntax. It even provides a web server that lints for errors and allows us to test our app locally.

\subsection{Flask}
Flask is a lightweight web framework for Python. Eventually, we will use Flask to provide our API endpoints. Our Flask-based backend will be responsible for querying our database, querying external REST APIs to collect data, and responding to requests to our API endpoints at poupon.me/api.

\subsection{Potential Future Tools}
\subsubsection{Insomnia REST Client}
In short, Insomnia is a cross platform application for running, organizing, and debugging HTTP requests/API calls. For our team, Insomnia is an attractive choice to make our API calls to fill our database due to its UI intuitive design. Furthermore, since our team uses a combination of Windows and Mac/Linux machines, having a cross platform application was convenient so that any member of the team could work with the tool. Overall, we plan to use this tool to populate our database/persistence layer, and potentially debug any HTTP requests that we test on our own website.

\subsubsection{Postman}
Postman is a REST Client similar to Insomnia. Much like Insomnia, it provides the same intuitive UI design and the same cross platform compatibility. The difference between Postman and Insomnia are minor UI design decisions; therefore, using one or the other is up to team/personal preference.

\end{document}

